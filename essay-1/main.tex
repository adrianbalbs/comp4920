\documentclass{article}
\usepackage{graphicx} % Required for inserting images
\usepackage[style=authoryear, backend=biber]{biblatex}

\addbibresource{bibliography.bib} %Import the bibliography file
\DeclareNameAlias{sortname}{family-given}
\renewbibmacro{in:}{}

\setlength{\parskip}{0.8em}

\title{COMP4920 Essay 1}
\author{Adrian Balbalosa, z5397730}
\date{September 2024}

\begin{document}

\maketitle

\section{Introduction}

\section{An Assessment of Kantian Ethics}
% TODO: Possibly add references here
Kantian Ethics is a deontological ethical theory which places emphasis on duty and
moral principles over consequences. Central to this framework is the notion of the
Categorical Imperative, which states that one should act according to the maxims which
can be universally applied. Another formulation of this is that individuals should treat
others as an ends, not just a means to an end. Kantian Ethics prioritises rationality and
personal freedoms, and argues that ethical actions should arise from a sense of duty and
adhering to moral law, rather than from emotional or situational considerations.

One of the strongest parts of Kantian Ethics is that it places emphasis on respect
for the individual. A fundamental aspect of Kantian Ethics is that 
humans should be respected because we are rational agent because we have the 
capacity for rational behaviour, and can be free from our impulses \parencite[p. 77]{bennet2015}.
This underpins the notion that humans ought to never be treated as means to our own devices,
because we are rational beings \parencite[p. 77]{bennet2015}. By ensuring that people are treated as
an ends, it upholds human dignity and rights.

However, there is a fundamental flaw of Kantian Ethics, in that when duties conflict, it is
not clear what action we are to take and how to resolve those dilemmas. A classical example used by
critics of Kantianism is the murderer at the door scenario, where the correct response is to respect
the autonomy of the murderer and tell the truth \parencite[p. 81]{bennet2015}. Lying could potentially 
save the life of someone, but we cannot lie as we would be disregarding the autonomy of the murderer, which
is paradoxical in nature. As a result of this, we are left to deliberate with difficult ethical decisions in
a complex situation like this.
\section{The Applicability of Kantian Ethics to Automated Ethics}

An opportunity which Kantian ethics presents is that it has been shown to be computationally
tractable. \textcite[p. 16]{singh2022} argues that "Kantian ethics is more natural to formalise" compared
to other ethical theories, as "the Formula of Universal Law evaluates the form and structure of an agent's
maxim" and requires less knowledge about the "state of affairs" or "moral character". This argument is 
further solidified through their implementation of an AMA that can successfully evaluate certain
ethical scenarios, like the nature of joking and lying \parencite[6--7]{singh2022}. It is noted that
there are still limitations to this implementation because the inputs and outputs require a specific
structure \parencite[p. 10]{singh2022}. Despite this limitation, we can observe that the use of Kantian
ethics makes the implementation of AMAs feasible.

A risk of the application of Kantian ethics to automated ethics is that artificial moral agents lack
genuine autonomy or consciousness. Kant states that transcendental freedom is "fundamental requirement of morality"
\parencite[p. 142]{mannananth2021}. That is in order to be considered a moral agent, a rational being should 
have the capability of acting autonomously rather than being controlled from external influences. \textcite[p. 149]{mannananth2021}
argue that "AI Systems are deterministic models of agency that do not exceed its initial programming". Since AMAs are
only able to act within the bounds of their programming, AMAs do not possess any consciousness or autonomy. The actions
of AMAs are mechanically driven, rather than driven by a sense of rationality. This could potentially lead to the
dehumanisation of the moral decision-making process, since AMAs lack the capacity for free and rational choice, a fundamental
Kantian value.

Another risk that is present is that an over-reliance on artificial moral agents would have the potential to diminish human engagement
with moral responsibility. \textcite[p. 146]{mannananth2021} claim that "AI's moral deeds are not generated from
the 'freedom of will' and the sense of 'duty' itself", but are generated by the programmer's command. The 
implications of this are even when these machines are programmed to follow ethical guidelines, they perform
them without actually understanding the moral reasoning behind them. Thus, the actions of AMAs are considered to
lack moral worth and are considered amoral. Through relying on AI to make our moral decisions, we risk reducing moral
actions to a mechanical process which lack the moral depth of deliberation. \textcite[p. 148]{mannananth2021} also state
that an AI agent "works according to hypothetical rules" rather than follow the categorical imperative, since their
actions are not performed out of a sense of duty, which is a central part of the categorical imperative. Hence, these
machines follows rules that are conditional rather than universal. The implications of this are that AI machines are
unable to understand the universal nature of the categorical imperative. Thus, with over-reliance on these systems,
we may become accustomed to following the rules without engaging with the deeper moral reasons that are essential for moral
action. 

\section{Conclusion}
While Katian ethics has shown to be an ethical theory which upholds human dignity and equality,
it fails in more complex situatons where maxims may clash against each other. The application
of Kantian ethics to automated ethics presents some interesting opportunities. One of them 
being that Kantian ethics is generally simpler to automate, as the categorical imperative provides
an algorithmic process for making ethical judgements. However, there are still some risks to consider
regarding its application. The first being the lack of genuine autonomy and over-reliance on these systems
leading us to de-value the moral decision-making process and make us accustomed to following the rules without
engaginng in deeper moral deliberation.

\printbibliography[title={References}]
\end{document}
