\documentclass{article}
\usepackage{graphicx} % Required for inserting images
\usepackage[style=authoryear, backend=biber]{biblatex}
\addbibresource{bibliography.bib} %Import the bibliography file
\DeclareNameAlias{sortname}{family-given}
\renewbibmacro{in:}{}

\title{COMP4920 Essay 1}
\author{Adrian Balbalosa, z5397730}
\date{September 2024}

\begin{document}

\maketitle

\section{Introduction}

\section{An Assessment of Kantian Ethics}
% TODO: Possibly add references here
Kantian Ethics is a deontological ethical theory which places emphasis on duty and
moral principles over consequences. Central to this framework is the notion of the
Categorical Imperative, which states that one should act according to the maxims which
can be universally applied. Another formulation of this is that individuals should treat
others as an ends, not just a means to an end. Kantian Ethics prioritises rationality and
personal freedoms, and argues that ethical actions should arise from a sense of duty and
adhering to moral law, rather than from emotional or situational considerations.

One of the strongest parts of Kantian Ethics is that it places emphasis on respect
for the individual. A fundamental aspect of Kantian Ethics is that 
humans should be respected because we are rational agents,
in that we have the capacity for rational behaviour, free from our impulses \parencite[p. 77]{bennet2015}.
This underpins the notion that humans ought to never be treated as means to our own devices,
because we are rational beings \parencite[p. 77]{bennet2015}. By ensuring that people are treated as
an ends, it upholds human dignity and rights.

However, there is a fundamental flaw of Kantian Ethics, in that when duties conflict, it is
not clear what action we are to take and how to resolve those dilemmas. A classical example used by
critics of Kantianism is the murderer at the door scenario, where the correct response is to respect
the autonomy of the murderer and tell the truth \parencite[p. 81]{bennet2015}. Lying could potentially 
save the life of someone, but we cannot lie as we would be disregarding the autonomy of the murderer, which
is paradoxical in nature. As a result of this, we are left to deliberate with difficult ethical decisions in
a complex situation like this.
\section{The Applicability of Kantian Ethics to Automated Ethics}
\section{Conclusion}
\printbibliography[title={References}]
\end{document}
